\documentclass{report}
\usepackage{graphicx}
\usepackage{xepersian}
\usepackage{geometry}
\settextfont[Scale=1.2]{XB Zar}

% section numbering
\renewcommand{\thesection}{\arabic{section}}
\renewcommand{\thesubsection}{\thesection.\arabic{subsection}}
\renewcommand{\thesubsubsection}{\thesection.\arabic{subsection}.\arabic{subsubsection}}

\title{ 
\begin{normalsize} به نام خدا \end{normalsize}
\\[8cm]
طراحی نقشه ساختمان با استفاده از پردازش تکاملی
\\[2cm]
}
\author{علیرضا نوریان
\\
\\ \small دانشگاه علم و صنعت ایران
\\ \small noorian@comp.iust.ac.ir
}

\begin{document}
\maketitle

\tableofcontents

\begin{abstract}

\end{abstract}

\section{مقدمه}
زندگی مدرن برای بشر مزایا و البته معایب بسیار زیادی به همراه آورده است. بسیاری از مشکلات حاصل از زندگی نو ریشه‌های عمیق فرهنگی دارند و شاید پاسخهای متناسب با ارزشهای دنیای جدید، برای آنها مناسب نباشند ولی در همه موارد این حرف صحیح نیست. پروژه‌ی پیش‌رو نمونه‌ای از حل مشکلات دنیای نو با استفاده از راهکارهای آن است. به طور مختصر می‌توان گفت هدف این پروژه فراهم کردن خانه‌های عملکردگرا برای سازنده‌ی مقتصد است که البته خریدار محکوم به زندگی در دنیای جدید از آن نفع خواهد برد.
به وجود آمدن مفهوم «بازار مسکن» را می‌توان از پیامدهای این زندگی جدید دانست. بازار مسکن به عنوان یکی از بازارهای پر سود موجب رشد و برخاستن جماعت «بساز و بفروش»ها شده است و این یعنی پایین آمدن کیفیت محل سکونت، کم شدن مساحت و ... که بدون تردید تاثیر بسیار بدی روی فرهنگ جامعه گذاشته است. این پروژه با هدف فراهم کردن نقشه‌ی بهینه ساختمان از جهات عملکردی انجام شده است و البته تا رسیدن به این مقصود فاصله‌ی بسیار زیادی دارد. در واقع ورودی نرم‌افزار تهیه شده در این پروژه قیدهای ساختمان از نظر طراح آن و خروجی نقشه‌ی ساختمان است.
«تولید نقشه‌ی بهینه‌ی ساختمان» جزء مسائل بهینه‌سازی محسوب می‌شود و راه حل آن کاملا وابسته به نحوه مدل‌سازی ساختمان است. در این پروژه ارائه‌ی ساختمان به گونه‌ای انجام گرفته که مساله با روشهای «پردازش تکاملی» قابل حل باشد. در ادامه پس از بررسی کوتاهی در مورد پردازش تکاملی به تعریف مساله و پاسخ داده شده به آن می‌پردازیم.

\section{پردازش تکاملی}
\subsection{الگوریتم ژنتیک}
\subsection{استراتژی تکاملی}
\subsection{روشهای چند هدفه}
\subsection{روشهای ترکیبی}

\section{تعریف مساله}
کاربر نرم‌افزار علاقه دارد با فشار دادن یک دکمه‌ی نقشه‌ی ساختمان را با جزئیات کامل دریافت کند. افق این پروژه نیز به همین شکل ترسیم شده است؛ اما آغاز راه باید کمی ساده‌تر تصور شود. در این مرحله از سیر پروژه، زمین ساختمان به صورت یک مستطیل در نظر گرفته شده و کاربر باید ابعاد آن را مشخص کند. همچنین نرم‌افزار هیچ شناختی از اتاقها ندارد و کاربر باید مساحت و یا ابعاد هر کدام از آنها را تعیین کند. قرار دادن اتاقهای مستطیل‌شکل در فضا موجب به وجود آمدن فضاهای خالی می‌شود؛ فضای دسترسی نامی است که برای این فضاهای باقیمانده در نظر گرفته شده و قابل استفاده کردن آنها از اهداف چالش برانگیز پروژه است. باید به محدودیتهای کنونی پروژه، فرض مسکن بودن ساختمان را نیز اضافه کرد. در واقع درک نرم‌افزار از فضای دسترسی، یک هال بزرگ متصل به حداقل راهروهای ممکن برای ایجاد دسترسی به اتاقها است.
خانه‌ی مطلوبی که قرار است به عنوان خروجی ارائه شود باید قابل سکونت باشد؛ یعنی علاوه بر انطباق با خواسته‌های کاربر از نظر ابعاد، دسترسی به همه‌ی اتاقهای آن ممکن باشد. البته کاربر وظیفه‌ی انتخاب نحوه دسترسی را بر عهده دارد و می‌تواند تعیین کند که مثلا مستطیل مربوط به آسانسور تنها از مستطیل راه‌پله قابل دستیابی است. 
آخرین وظیفه‌ی کاربر تعیین نوع نورگیری ساختمان در ضلعهای مختلف آن است. شناخت نحوه گرفتن نور توسط ساختمان و ورودی گرفتن اولویت نوری میان اتاقها از کاربر به نرم‌افزار امکان طراحی خانه‌ای با الگوی نورگیری مناسب را می‌دهد. امید است که در نگارشهای بعدی نرم‌افزار قابلیت تصمیم‌گیری در مورد نیاز خانه به نورگیر و مشخصات آن را کسب کند. ارزیابی یک ساختمان و به طور خاص یک خانه جنبه‌های بی‌شماری دارد که حضور در مراحل آغازین بهانه‌ای قابل قبول برای صرف نظر کردن از آنهاست.

\section{نحوه بیان مساله}
حل این مساله با فرایند تکامل نیازمند توصیف پاسخ آن در قالب تعداد محدودی عدد حقیقی است که به مجموعه‌ی آنها ژنوم می‌گوییم. معنی این اعداد هرچه از هم مستقل‌تر باشد، جستجو در فضای حل مساله آسان‌تر انجام می‌گیرد؛ از این‌رو هر اتاق ساختمان را با چهار مؤلفه فاصله افقی و عمودی از مبدا و طول و عرض آن نمایش می‌دهیم. با این نحوه‌ی ارائه اگر همه‌ی اعداد ژنوم مثبت باشند، می‌توان آن را ژنومی معتبر دانست. اگرچه این سطح از اعتبار تضمین نمی‌کند که اتاقها با هم تداخل نداشته باشند، اما می‌توانیم مطمئن باشیم که اتاقی با طول و عرض منفی نداریم.

\section{عملگرها}
عملگرهای فرایند تکامل باید با ژنومی که برای توصیف پاسخها استفاده می‌شود، سازگار باشند. بعضی از انواع عملگرها به طور عام طراحی شده‌اند و این پروژه نیز از آنها بی‌بهره نبوده است. اما علاوه بر عملگرهای عام، عملگرهایی نیز به طور خاص برای ژنوم توصیف کننده ساختمان تعریف شده‌اند. نکته‌ی مشترک در این عملگرها، نگاه به اتاقها با عبور از اعداد توصیف کننده‌ی آنهاست. در واقع عملگرهای عام به هر عدد موجود در ژنوم به طور مستقل نگاه می‌کنند در حالی که ما به دنبال عملگرهایی هستیم که در آنها اتاقها موضوعیت داشته باشند.
برای نمونه در نوع خاصی از عملگر ادغام هر ویژگی فرزند به طور احتمالی از یک والد انتخاب می‌شود. این عملگر با انگیزه‌ی ایجاد فرزندی با ویژگی‌های والدها طراحی شده و انتظار می‌رود فرزند مربوط به والدهای بهتر، بخشی از شایستگی‌های هر والد را به همراه داشته باشد. حتی در یکی از شکلهای اجرا، ممکن است تعداد زیادی والد همزمان برای تولید تعداد زیادی فرزند مورد استفاده قرار گیرند. تاثیر این عملگر روی ژنوم توصیف کننده ساختمان به نظر کمی متفاوت با هدف طراحی آن می‌رسد؛ چراکه شایسته بودن والد بیش از آنکه به هر عدد وابسته باشد، به هر اتاق (چهار عدد متوالی) وابسته است و احتمال بسیار زیادی وجود دارد که فرزند از هر دو والد شایستگی کمتری داشته باشد. این پدیده با بی‌اثر کردن عملگر ادغام، علاوه بر کند کردن فرایند تکامل، موجب کاهش گوناگونی در پاسخها و به عبارتی بروز همگرایی در آنها می‌شود. در توضیح این ادعا باید گفت که با کم‌رنگ شدن نقش عملگر ادغام، همه‌ی وظیفه‌ی ایجاد تنوع روی دوش عملگر جهش می‌افتد و پاسخهای نسبتا بد در مقابل پاسخهای تولید شده از جهش‌های مختلف یک جواب بهتر، حذف می‌شوند. در ادامه به عملگرهایی که برای این مساله خاص طراحی شده‌اند، توجه می‌کنیم و نتایج مربوط به این پدیده را در بخش نتایج مربوط به روشهای مختلف، مشاهده می‌کنیم.

\subsection{ادغام اتاقها}
این عملگر از دو والد که از مرحله انتخاب عبور کرده‌اند، دو فرزند ایجاد می‌کند. هر اتاق در هرکدام از فرزندها با احتمال مشخصی متعلق به یکی از دو والد است. برای نمونه ممکن است فرزند تولید شده آشپزخانه را از یک والد و اتاق خواب را از والد دیگر به ارث ببرد.

\subsection{جهش تعویض مکان دو اتاق}
این عملگر روی فرزندان حاصل از عملیات ادغام، اعمال می‌شود و همانطور که اشاره شد به معنی مجموعه‌ی اعدادی که یک اتاق را نمایش می‌دهند توجه دارد. این عملگر مرکز قرار گیری دو اتاق را که به طور اتفاقی انتخاب شده‌اند، با هم عوض می‌کند. در مراحل آغازین فرایند تکامل (زمانی که پاسخها نقصهای متعددی دارند) و همچنین هنگامی که شرطهای مربوط به ابعاد برای دو اتاق انتخاب شده مشابه هم است، تاثیر بهتری را می‌توان از چنین عملگری مشاهده کرد. 

\section{تابع ارزیابی}
از ژنوم صحبت کردیم و اینکه نقشه‌ی ساختمان را با تعدادی عدد حقیقی نمایش می‌دهیم. فرایند تکامل برای رسیدن به هدف نیاز به یک راهنما دارد و این راهنما باید نتایج فرایند را به سمت ساختمان هدایت کند. نقشه‌ی ساختمان ویژگی‌های زیادی باید باشد، تابع ارزیابی باید هر کدام از این ویژگی‌ها را با استفاده از معیاری کمّی بسنجد تا تحقق آنها در پاسخ ممکن باشد.
در این پروژه محدودیت زمانی مانع از تعریف معیارهای بسیار پیچیده برای ارزیابی ویژگی‌ها ساختمان شد. از این رو نتایج فرایند تکامل را نمی‌توان خانه‌های مناسبی برای زندگی دانست؛ اگرچه شباهت زیادی به آنها دارند و یا حتی در مواردی قابل سکونت هم هستند. برای نمونه ساختمانهای خروجی قابل اجرا هستند یعنی اتاقها با هم تداخل ندارند و حتی دیوار بین آنها هم مورد توجه قرار گرفته؛ اما روشن است که انتظار ما از خانه خیلی بیشتر از قابل اجرا بودن است.
بعضی از معیارها با استفاده از تحلیل هندسی هر اتاق و یا هر دو اتاق انجام می‌شوند. چنین معیارهایی معمولا ویژگی‌های ساده‌ای را مورد بررسی قرار می‌دهند و طبیعتا ابهام کمتری دارند. برای نمونه ما انتظار داریم ابعاد اتاق خواب در خانه مقادیر مشخصی باشد، معیار متناظر با این ويژگی مورد انتظار، به صورت سنجش مساحت هر اتاق به طور جداگانه تعریف می‌شود. اما بعضی از ویژگی‌ها نیاز به بحث و بررسی بیشتری دارند.
غیر از این معیارهای ساده، معیارهای دیگری هم وجود دارند که به فضای خالی میان اتاقها را ارزیابی می‌کنند. نحوه توصیف این فضا و محاسبه‌ی آن یکی از چالشهای این پروژه است که در توضیح بیشتری در مورد آن می‌دهیم.

\subsection{فضای دسترسی}
ارضای معیارهای مربوط به فضای خالی میان اتاقها، کاملا وابسته به تعریفی است که از این فضا ارئه می‌کنیم. در این پروژه تعریف فضا به صورت قرار دادن مستطیلهای فرضی در پاسخ صورت گرفته است. در واقع الگوریتمی ابتکاری برای قرار دادن بزرگترین مستطیلهای ممکن برای پوشاندن فضای خالی تعریف شده است. مستطیلهای بدست آمده از این الگوریتم ماده خام برای انجام ارزیابی‌های فضایی هستند. بزرگترین و متناسب‌ترین مستطیل از میان این مستطیلها به عنوان هال انتخاب می‌شود و بقیه نقش راهرو و پیش‌فضا را بازی می‌کنند. در تعریف پیش‌فضا باید گفت که ورودی بعضی از اتاقها مثل حمام بهتر است که در تماس مستقیم با فضای دسترسی اصلی نباشد. پوشش فضایی به گونه‌ای انجام می‌شود که فضاهای خرد انتخاب نشده و فقط فضاهای قابل استفاده انتخاب شوند. 

\begin{figure} \caption{\label{fAccessSpace} نمایش فضای دسترسی بین اتاقها} \end{figure}

\subsubsection{روش محاسبه فضای دسترسی}
فضای خالی میان اتاقها باید با بزرگترین مستطیلهای ممکن پوشانده شود. این مستطیلها می‌توانند روی هم قرار بگیرند ولی طول یا عرض آنها نباید کوچکتر از مقدار مشخصی باشد. برای پیدا کردن مستطیلهای پوشاننده، ابتدا شبکه‌ای از نقاط را روی صفحه قرار می‌دهیم و سعی می‌کنیم بزرگترین مستطیلی را که از هر نقطه عبور می‌کند، محاسبه کنیم. برای پیدا کردن بزرگترین مستطیل عبور کننده از یک نقطه، به فاصله‌ي آن نقطه تا مستطیلهای مجاور آن نگاه می‌کنیم و به این ترتیب سعی می‌کنیم طول و عرض مستطیل را پیدا کنیم.
در نهایت بزرگترین مستطیل از میان مستطیلهای مشخص شده را به عنوان فضای اصلی دسترسی تعیین می‌کنیم. این مستطیل در خانه‌ی مسکونی همان اتاق هال است و هر مستطیلی که در تماس با آن نباشد، مکانی غیر قابل دسترسی فرض شده و حذف می‌شود.
می‌توان گفت این روش به دنبال بررسی همه‌ی مستطیلهای ممکن برای پوشاندن فضای خالی است و قرار دادن شبکه‌ای از نقاط روی صفحه موجب می‌شود که مساله از فضای اعداد پیوسته به اعداد گسسته انتقال یابد. البته برای حل این مساله احتمالا راههای بسیار بهتری از آنچه که توضیح داده شد، وجود دارد. امید است که در آینده این روش مورد بازبینی و اصلاح قرار گیرد.

\subsection{معیارها}

\subsubsection{تداخل}
روشن است که مستطیلهای نمایانگر اتاق نمی‌توانند با هم تداخل کنند و همچنین هیچکدام از آنها نمی‌توانند از فضای نقشه خارج شوند. بررسی این شرطها در معیار تداخل و با شکل میل به کاهش تداخلها انجام می‌شود.

\subsubsection{مساحت}
سیاست اصلی بزرگ شدن فضاهای دسترسی ارزشمند و از همه مهمتر فضای دسترس اصلی (هال) است. نکته‌ی دیگر کم شدن تعداد فضاهای مربوط به راهرو و پیش‌فضاست. در واقع کم شدن این فضاها به معنی بزرگ شدن فضای دسترسی اصلی و تمیز شدن آن است.

\subsubsection{دسترسی}
هر کدام از اتاقها باید از طریق فضای دسترسی قابل دستیابی باشند. دسترسی در یک ساختمان یعنی تماس میان دو فضا به اندازه‌ی حداقل یک «در» که این دو را به هم متصل کند. به طور کمّی‌تر می‌توان گفت باید فاصله‌ی میان اتاق تا نزدیکترین مستطیل پوشاننده حداقل باشد.

\subsubsection{نور}
اضلاع مختلف در یک ساختمان بهره‌های متفاوتی از نور دارند. برای نمونه در خانه‌های معمول شهری بعضی از اتاقها نور مستقیم و بعضی نور آسمان (ضعیف) دریافت می‌کنند. از طرفی نیاز اتاقها نیز با هم متفاوت است. برای نمونه هال و اتاقهای خواب به نور بیشتری نیاز دارند و در مقابل سرویسها و حمام نیازی به نور ندارند. کمی کردن این معیار هم نیاز به تقریبهای مهندسی دارد. 

\section{پیاده‌سازی}
هسته‌ی اجرا کننده فرایند تکامل در این نرم‌افزار در چهارچوب ParadisEO نوشته شده است. این چهارچوب امکانات بسیار زیادی را در اختیار توسعه دهنده قرار می‌دهد و از چهار بخش اصلی تشکیل شده است ...

\section{نتایج روشهای مختلف}
در مدت انجام این پروژه روشهای زیر برای رسیدن به مجموعه جواب خوب و متنوع آزمایش شدند که در ادامه نتایج هر روش را بررسی می‌کنیم.

\subsection{الگوریتم ژنتیک}
اجرای الگوریتم ژنتیک با عملگرهای تعریف‌شده نتایج چندان جالبی به همراه ندارد. در این حالت اجرای الگوریتم نمی‌تواند تابع ارزیاب پیچیده (شامل همه‌ی موارد گفته شده) را ارضا کند و ادامه‌ی فرایند تکامل منجر به تولید پاسخ نمی‌شود. نتایج اجرا با نسخه‌ی ساده شده‌ی تابع ارزیاب (شامل مساحت و تداخل) منجر به تولید خروجی‌های نسبتا قابل قبولی می‌شود که البته چندان متنوع نیستند.

\subsection{استراتژی تکاملی}
اجرای این روش بدون استفاده از عملگرهای تعریف شده مخصوص مساله موجب تولید جواب بسیار قابل قبول شد. این نتیجه ممکن است کمی عجیب به نظر برسد و ضرورت تعریف عملگرهای مناسب را زیر سوال ببرد. نتایج و بررسی‌های بعدی نشان داد که اجرای این روش برای رسیدن به پاسخ، عملا منجر به اجرای یک جستجوی موضعی می‌شود و مستقل از تعداد جمعیت آغاز کننده فرایند تکامل، این فرایند به یک پاسخ بهینه‌ی موضعی همگرا می‌شود. این نکته‌ی بسیار مهم باعث شد که گزینه‌ی جستجوی موضعی را مورد بررسی قرار دهیم.

\subsection{روشهای چند هدفه}
اگرچه انتظار از اجرای این روشها رسیدن به تنوع بالا در جواب است، اما در این محله نیز خروجی‌ها تشابه زیادی به اجرای الگوریتم ژنتیک ساده دارند. البته با استفاده از تابع ارزیاب ساده تنوع جوابها در این روش تا حدی بهتر است.
نکته‌ی بسیار مهم در این روش و روش الگوریتم ژنتیک معمولی آن است که با افزایش تعداد فرزندان تولید شده و همچنین بالا بردن نقش عملگر جهش می‌توان به خروجی‌های بهتر و البته تقریبا همگرا دست یافت. ناگفته پیداست که این عوامل فرایند تکامل را به جستجوی محلی تبدیل می‌کنند.

\subsection{روشهای ترکیبی}


\section{نتیجه‌گیری}

\section{کارهای آینده}

\end{document}
